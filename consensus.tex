%!TEX root = synchrony.tex
\section{An Upper Bound For Consensus}\label{sec:consensus}

In this section, we establish an upper bound for binary consensus with oral messages. 
In binary consensus, processes do not need to agree on the initial value of each process. It is enough that they agree on some value to reach agreement.
%So in our setting, binary consensus is easier than interactive consistency.

\begin{theorem} \label{thmConsensus}
It is possible to achieve binary consensus in an  $(n,m,d)$-system with oral messages if  $n>2m+d$.
\end{theorem}

We give a new algorithm OMC which is slightly different from OMIC. OMC achieves binary consensus.
The basic idea in OMC consists in adding a \tmem{feedback step} to inform the transmitter about the guessed value other processes agreed on. The agreement is also based on majority selection. When $n$ is only greater than $2m+d$ (not max$\{2m+d,2d+m\}$), the majority value in algorithm OMIC may be several different values.
So we assume a selection function \tmem{Major} to fix such situations. 
\tmem{Major} takes a list as input. 
The requirement we make for this function is that it alway returns the majority value in the input list. 
If there are several values with the same highest count in the input, \tmem{Major} will return one of them. 
The function is deterministic. The same input list (the order of the elements can be different) leads to the same output. 
Each process $i$ maintains a list $V_i$ to record the
guessed values for all processes.

{\namedalgorithm{OMC}{OMC($0$):
\begin{enumeratenumeric}
  \item Every process broadcasts its initial value.
  
  \item Every process, for example $i$ uses the value it receives from process $j(\neq i)$ as
  $v_{j}$, or uses $\varnothing$ if it receives nothing. Let $v_i$ denote the initial value of $i$. 
  \item Take $Major([ v_{1} ,\ldots ,v_{n} ])$ as output of $i$.
\end{enumeratenumeric}

OMC($k$), $k>0$:
\begin{enumeratenumeric}
  \item Each process acts as a transmitter and sends its value to other
  $n-1$ processes (receivers).
  \item Every receiver process uses the value it gets in step $1$ as initial
  value, and executes OMC($k-1$) for one time to decide a value as a guessed initial value for the transmitter. After this, receiver $i$ appends the guessed values into $V_i$. Note that there are $n-1$ transmitters different from $i$, so $V_i$ now has $n-1$ values.
  \item \tmem{Feedback step}:
  every receiver $i$ sends its guessed initial value for transmitter $j$ to $j$. $j$ add the majority of the received values in this round into $V_i$ as the replacement of its initial value.
  \item Process $i$ takes $Major(V_i)$ as its output.
\end{enumeratenumeric}}}

As we already showed in Section \ref{oralIC}, we cannot guarantee interactive consistency with OMIC with only $n>2m+d$. 
However we ensure interesting property in OMC with respect to $n>2m+d$. 
More specifically, when $n>2m+d$, we can prove that the elements in list $V_i$ are the same for different $i$, but the values in $V_i$ are not always equal to the initial values of the transmitters. 
We show this by induction as in Section \ref{oralIC}. Due to space limitation, the proof is moved to Appendex.

\iffalse
\begin{lemma}\label{cons:2}
If $n\geqslant 2m+2d$, then  OMC(1) achieves  binary consensus in $2$ rounds in an $(n,m,d)$-system.
\end{lemma}

\begin{proof}
As we have shown in Lemma~\ref{basicCase}, the guessed value in Step $2$ 
is the same as the initial value of transmitter when $n \geqslant 2m+2d$. 
As the output of each process is the result of applying \tmem{Major} to the list consisting of the initial values the properties of binary  consensus are satisfied.
\end{proof}

\begin{lemma}
For any $k\geqslant 1$, if $n>2m+k$, OMC($k$) ensures that the guessed value in Step $2$ for a correct transmitter is the initial value of the transmitter.
\end{lemma}

\begin{proof}
In the case $k=1$, the guessed value is the output of OMC(0) and  it is equal to the initial value of the correct transmitter since $n-1>2m$. Suppose the lemma is proved for $k-1$. Let us prove it for $k$.

In OMC($k$), the correct transmitter $i$ first sends its value to the other $n-1$ receivers. 
By induction hypothesis, the initial value of correct process will be guessed  correctly in OMC($k-1$). 
Since there are at least $n-1-m(>m)$ correct receivers, the decision values of all the receivers are the initial value of the transmitter.
\end{proof}

\begin{lemma}\label{cons:corelemma}
  For any $k \geqslant 1$, OMC($k$) achieves binary consensus if $n>\max\{2m+k,2m+2d-k\}$.
\end{lemma}

Note that this lemma is different from Lemma~\ref{mainlemma} in the sense that $k\leqslant m$ is not required.

\begin{proof}
We proof this lemma by induction on $k$. The basic case $k=1$ is the same as in Lemma~\ref{cons:2}.
Hence, we suppose the lemma is proved for $k-1$, and we prove it for $k$.

If the transmitter is correct, every receiver decides the same guessed value in step $2$ for $n>2m+d$. 
If the transmitter is faulty, then
by induction hypothesis, every receiver  guesses the same value for every transmitter. 
In the \tmem{feedback step}, the transmitter will receive $n-1$ copies of the guessed value. 
Since there are at most $m$ faulty receivers, and $n-1>2m$, the transmitter 
also gets the same guessed value in Step $3$. 
So the final list $V_i$ is the same for different $i$ proving the agreement property. 

Now we consider  the validity property.
Suppose all input values are the same $v(\in \{0,1\})$. 
Since $n>2m+k$, by the last lemma, the guessed value for correct transmitters is the real initial value $v$. So there are at least $n-m (> m)$ elements in $V_i$ with value $v$. Therefore, \tmem{Major}($V_i$) is $v$. 
The validity property is also proved.
\end{proof}

With this core lemma, it is easy to prove Theorem~\ref{thmConsensus}.
\begin{proof}{(Proof of Theorem~\ref{thmConsensus})}
We prove the theorem by taking $k$ equal to $d$ in Lemma~\ref{cons:corelemma}.
\end{proof}
\fi

\begin{remark}
Though we have proved an upper bound for consensus, we know that this bound is not tight. 
For example, binary consensus is achievable with $m=1$ and $n=2m+d$. 
However, we conjecture that $2m+d$ is the tight bound if $m \geqslant n/3$.
\end{remark}
%%% Local Variables: 
%%% mode: latex
%%% TeX-master: synchrony
%%% End: 
