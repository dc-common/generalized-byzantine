%!TEX root = synchrony.tex
\section{Model and Definitions}
 
We consider a synchronous message-passing distributed system $P$ of $n$ processes. 
Each process is
identified by a unique id $p \in \{ 1, \ldots ,n \}$.  
As in {\cite{lamport1982byzantine,toueg1984simple}}, a
{\tmem{synchronous computation}} proceeds in a sequence of $rounds$.    
 The nodes communicate
with each other by sending messages round by round within a fully
connected 
point-to-point network.
In each round, every process first sends at most one message to every other process, possibly to all processes, 
and then $p$ receives the messages sent
by other  processes. The communication channels are authenticated, i.e., the sender is known to the recipient. 

Each process has
an input register with its initial value from some domain $D$, and an
output register  
which records the outcome of the computation.
%The domain of input values 
Note that the output register can be written at most once. When a
process writes its output register, we say that the process decides.
We model an algorithm as a set of deterministic automata, one for
every process in the system. 
Thus, the actions of a process are entirely determined by the
algorithm, the initial value and the messages it receives from others. 
In this paper we assume that processes always follow their protocol.


\subsection{Failure model}
%\vspace{1em}\noindent{\bf Failures. }
In short, a faulty process $p$ may lie to other processes: even if $p$ follows its code, $p$ can send to a subset of processes Byzantine messages, i.e., messages that result from $p$ not following the protocol.

More precisely, we assume an adaptive (or dynamic) adversary
which introduces Byzantine faults in transmission: these faults may
only come from faulty processes. Here, up to $m$ of the
processes are 
faulty and controlled by the adversary. In each round, the adversary chooses
up to $t$ communication links from each faulty process that
will carry Byzantine messages. 


An instance of our model of $n$ processes with $m$ 
faulty processes such that in each round up to $t$ communication faults
on links from faulty processes may occur will be called an {\tmem{$( n,m,t )$-system}. For the results established in this paper, we always assume $m>0$, $t>0$, and $n > 
max\{m,t\}$. But the model itself does not preclude other cases. We also assume $n>1$, the case $n=1$ being trivial. 

\vspace{1em}
 We consider two authentication cases:
 \emph{oral messages} and \emph{signed messages}.
%  depending on the ability t
% We consider here the ability or not of authenticated messages with
% unforgeable signature~\cite{}
% Concerning Byzantine faults, an important  is the ability of having 
% the processes to send 
% unforgeable signed messages.
Following~\cite{lamport1982byzantine,srikanth1987optimal}, with oral messages, the
sender is always
authenticated by the receiver. But contrary to messages with
unforgeable signatures, a process $q$ may make process $p$ believe that $q$ has received
message $m$ from process $r$ even if it is not true.
%Another important assumption is whether or not the model supports signatures. 
% We use oral messages and signed messages to model these two situations, 
% as in\cite{lamport1982byzantine,ST87}. An oral message is the one
% without signature. 
% and whose the  contents are completely controlled by the sender. 
For a signed message,  the sender attaches its
signature to the messages. 
A signed message satisfies the two following properties: 

\begin{enumeratealpha}
  \item A correct process's signature cannot be
    forged and any alteration of the content of its signed messages
    can be detected. 
  \item Any process can verify the authenticity of a process's
    signature. 
\end{enumeratealpha}
Note that the signature of a faulty process can be forged by
another faulty process. 

\subsection{Full information protocols}
%\vspace{1em}\noindent{\bf Full information protocols.}
To prove our impossibility results, we consider full information protocol as in 
{\cite{fischer1982lower,lamport1982byzantine,lynch1996distributed}. In a full information
protocol, every process transmits to all processes in each round everything it knows
about all the values sent by other processes in the previous round.

We use $P^{l:k}$ to denote the set of strings of symbols in $P$ of length at least $l$ 
and at most $k$, $P^{+}$ to denote nonempty strings of symbols in $P$
and $P^{*}$ to denote all the strings including the empty one.

 A {\tmem{$k$-round scenario}} (for $(n,m,t)$-system $P$) describes an
 execution of the protocol. Intuitively  $\sigma$ describes a 
 communication scheme admissible for a $(n,k,m)$ system.
 It gives the initial value of each process and the communication scheme.
It captures the outcome of a $k$-round full information exchange.
Given scenario $\sigma$, 
$\sigma ( p_{1}
p_{2} \ldots p_{k} )$ is intended to represent the value $p_{k-1}$
tells $p_k$ that $p_{k-2}$ tells $p_{k-1}$ ... that $p_1$ tells $p_2$
is $p_1$'s initial value. 

 More specifically,  a  {\tmem{$k$-round scenario}} $\sigma$
  is a mapping :$P^{1:k+1}
\rightarrow D$, such that:
\begin{itemize}
\item
For a string $p$ of length $1$, $\sigma(p)$ is $p$'s initial value. 
\item
There is a partition of $P$ into two sets $R_{\sigma} $, the set of
correct processes,  and $U_\sigma$, the set of faulty ones
such that:
\begin{itemize}
\item
$|U_\sigma| \leq m$ (and then $|R_\sigma|\geq n-m$)
\item
for every  process $p\in R_{\sigma} : \sigma(w p q) = \sigma(w p)$ for all
$q\in P$ and $w\in P^{0:k-1}$,  
\item
for every process $p\in U_\sigma$,  for every round $j$ in $\{1,\ldots
, k\}$,  there is a set $T$ of at most $t$ processes such that for
all $q\in P\setminus T$  and for all
$w\in P^{0:k-1}$ we have $\sigma(w p q) = \sigma(w p)$.
\end{itemize}
\end{itemize}
Note that if  $\sigma(wpq) \neq \sigma(wp)$ for some strings $w$ of
length $l$, that means that $q$  receives a Byzantine message from $p$
in round $l$.



If $\sigma$ is a
{\tmem{$k$-round scenario}} and $p \in P$, $p$'s view of \ $\sigma$ is
the map $\sigma_p$ defined by 
$\sigma_{p} ( w ) = \sigma ( w p )$. 


Let $O$ be the set of possible outputs and $\mathcal{U}^{k}$ be the
set of mappings from $P^{k}$ into $D$.
% A  $k$-round algorithm $\cal A$ defined in a $(n,m,t)$ system
% may be defined  on the set of all scenario; namely as a set $\{ F_{p}
% : p \in P \}$ of functions, 
% where $F_{p} : \mathcal{U}^{k}  \rightarrow O$. 
%  $\cal A$ solves a problem 
%  if for each $k$-round scenario $\sigma$ and every processes $p
% \in P$,  $F_{p} ( \sigma_{p} )$ satisfies the specification of the
% problem.
Any $k$-round algorithm $\cal A$ defined in an $(n,m,t)$-system
may be defined  on the set of all scenarios; namely as a set $\{ F_{p}
: p \in P \}$ of functions, 
where $F_{p} : \mathcal{U}^{k}  \rightarrow O$. 
Then without loss of generality, we consider in the following only algorithms
defined this way on the set of all scenarios by a set $\{ F_{p}
: p \in P \}$ of functions.

 $\cal A$ solves a problem 
 if for each $k$-round scenario $\sigma$ and every process $p
\in P$,  $F_{p} ( \sigma_{p} )$ satisfies the specification of the
problem.




\subsection{Problem specifications}
%\vspace{2em}\noindent{\bf Problem specifications.}
We will first address the problem of {\tmem{interactive consistency}}~\cite{fischer1983consensus}. 
In this case, each process maintains an output register of $n$ entries. 
The $j$-th entry of the output register of a process $i$ contains the value decided by process $i$ for process $j$.
It is defined by the  two following properties:
\begin{itemizedot}
  \item {\tmem{Termination}}: Eventually, every process decides a value for
  each process.
  
  \item {\tmem{Validity}}: If process $p$ decides $v$ for process $q$, then
  $v$ should be the initial value of $q$.
\end{itemizedot}
As a remark, validity implies that all the processes decide the same
values.
Note that, contrary to classical models with faulty processes, here
even faulty processes have to decide. 

Let ${\cal A}= \{
F_{p} :p \in P \}$ be a $k$-round algorithm and the output is a vector that contains the  $n$ decided values then  $O$ is $(D\cup \{\bot\})^n$.
Then $\cal A$ solves  interactive
consistency if for each $k$-round scenario $\sigma$ and 
every process $p
\in P$,  $F_{p} ( \sigma_{p} )$ satisfies the Termination and Validity properties of interactive consistency, i.e.
for each $k$-round scenario $\sigma$ and all $p,r
\in P$ we have $F_{p} ( \sigma_{p} )[r] = \sigma ( r )$.


\vspace{1em}
We will also consider the {\tmem{binary consensus}} problem~\cite{fischer1983consensus}. 
In this problem, we assume that every process's input value is in $\{0,1\}$
and its output register contains $\bot$, 0 or 1. $\bot$ means that the
register has not yet written by the process. 
The value written in the output register is named the decision value.
The  {\tmem{binary consensus}} problem is defined by the three
following properties: 
\begin{itemizedot}
  \item {\tmem{Termination}}: Eventually, every process decides a value.
  \item {\tmem{Agreement}}: Every two processes decide the same value.
  \item {\tmem{Validity}}: If  all processes start with the same
    initial input, then every process should decide this value. 
\end{itemizedot}


Let ${\cal A}= \{
F_{p} :p \in P \}$ be a $k$-round algorithm and the output is in $\{0,1,\bot\}$.
Then $\cal A$ solves  binary consensus if for each $k$-round scenario $\sigma$ and 
every process $p
\in P$,  $F_{p} ( \sigma_{p} )$ satisfies the Termination, Agreement
and Validity properties of binary consensus, i.e.   
if for each $k$-round scenario $\sigma$, there exists $v$ the initial
value of some process such that  
for all processes $p$ and $r$,  we have $F_{p} ( \sigma_{p} ) = v$,
where $v$ is the initial value of some process.

%%% Local Variables: 
%%% mode: latex
%%% TeX-master: "synchrony"
%%% End: 
