%!TEX root = synchrony.tex
\section{Encompassing Crash Failures}\label{crash}
To simplify, we assume so far that faulty processes can be 
$d$-faulty, with $d < n-1$, excluding crash failures. 
Crash failures  can easily be taken into account by considering, in addition to $d$-faulty processes that may send up to $d$ wrong messages in each round, 
$c$-faulty processes that may crash in some round. If a
$c$-faulty process crashes in some round, some of its messages for this round
may be received and  others not. In the subsequent rounds however, no messages are 
sent.   Note that a process may be both $d$-faulty and $c$-faulty.

We call a $( n,m,d,c )$-system for a system of $n$ processes, with up to $m$
$d$-faulty processes and up to $c$ $c$-faulty processes.
% where $d<n-1$, for which $d$-faulty processes always send correct messages to
% a subset of processes. In this section we investigate the interactive consistency problem
% for this hybrid model with oral message.
% From the practical perspective, crash failure \cite{fischer1985impossibility} is typical and important. Though
% this failure can be modeled into the case $d=n-1$, it is too conservative. In
% this section, we discuss a new kind of hybrid model where $d$-faulty processes and
% crash-faulty processes occur simultaneously.
% {\tmem{Crash failures}} means a faulty processes might crash at some point,
% after which it sends no messages at all. In the hybrid model to be discussed,
% the system consists of $n$ processes, of which upto $m$ processes are
% $d$-faulty processes and upto $c$ processes might crash. An instance
% of this hybrid model is called $( n,m,d,c )$-system.
The \tmem{interactive consistency}  specification can be adapted as follows to the 
case of a  $( n,m,d,c)$-system:
  \begin{itemizedot}
    \item {\tmem{Termination}}: Eventually, every process that is not $c$-faulty
     decides a value for each process.
    
    \item {\tmem{Agreement}}: Every two processes that decide,
    decide the same value for each process.
    
    \item {\tmem{Validity}}: If some process $p$ decides $v\neq \bot $ for process $q$, then
           $v$ is the initial value of $q$ and if $p$ decides $\bot$
           for process $q$, then $q$ is $c$-faulty.
  \end{itemizedot}
Note that if $c=0$,  the problem specification is the same as
the one in Section~\ref{oralIC}.

% We focus on the situation
% where $d<n-1$, for which $d$-faulty processes always send correct messages to
% a subset of processes.
% In this section we investigate the interactive consistency problem
% for $( n,m,d,c)$-system with oral messages.


% Through this section, a $( n,m,d,c )$-system is supposed with oral
% messages. Based on the results
% for our generalized Byzantine model in the previous sections, we can have the
% following corresponding theorem.
As an extension of Theorem~\ref{consensusOral} we get:
\begin{theorem}\label{crashIC}
  Interactive consistency can be achieved in a $( n,m,d,c )$-system with oral
  message if and only if $n> \max \{ 2m+d,2d+m \} +c$.
\end{theorem}


% As the results in \cite{lamport1982crash,dolev1982polynomial,dwork1990knowledge} say, tolerating $c$ crash failures in the worst case
% must require at least $c+1$ rounds message-passing.
%  So we have to modify our previous algorithm to make it work.  In this section, firstly a modified version of OMIC (called OMWIC)
% is designed to reach a weak agreement, which can reduce
% the interactive consistency problem to the classical problem with only crash failures.
% Then we can use known algorithm in \cite{lamport1982crash,dolev1982polynomial,dwork1990knowledge} that only tolerates crash failures to reach interactive
% consistency.
% Our previous algorithm can be modified to tolerate crash failures.  
We first devise a modified version of our previous OMIC protocol (which we call OMWIC)
to solve a weaker version of interactive consistency (weak interactive consistency) where 
processes do not need to agree on the initial value  of $c$-faulty processes. 
The difficulty of the modified algorithm (OMWIC) is to select the  value to output under the interference of empty messages produced by $c$-faulty processes. 
In order to address this, we introduce a threshold parameter into the algorithm to change the majority function to filter  empty messages as well as faulty messages. This idea is novel and we believe it could be applied to the design of 
other distributed algorithms.  

Then we use a classical crash-tolerant synchronous algorithm \cite{lamport1982crash,dolev1982polynomial,dwork1990knowledge},  
in which we replace the send/receive operations by our weak interactive consistency algorithm.
Due to space limitations, the algorithms and proofs are moved to Appendix~\ref{app:thmCrash}.





%%% Local Variables: 
%%% mode: latex
%%% TeX-master: "synchrony"
%%% End: 
