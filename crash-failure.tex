\section{Extension to Crash Failures}
In the previous sections we assumed that faulty processes are only
$d$-faulty. In this section we extend our results to crash failures.
For this we consider two kinds of faulty processes: $d$-faulty processes as
defined in the previous sections may send in the rounds  up to $d$ wrong messages and
$c$-faulty processes that may crash in some round. If a
$c$-faulty process crashes in some round some of its messages for this round
may be sent and some other not and in all next rounds no messages from
this process are sent. 
Note that a process may be both $d$-faulty and $c$-faulty.

A $( n,m,d,c )$-system is a system of $n$ processes, with up to $m$
processes are $d$-faulty and up to $c$ processes are $c$-faulty.
% where $d<n-1$, for which $d$-faulty processes always send correct messages to
% a subset of processes. In this section we investigate the interactive consistency problem
% for this hybrid model with oral message.
% From the practical perspective, crash failure \cite{fischer1985impossibility} is typical and important. Though
% this failure can be modeled into the case $d=n-1$, it is too conservative. In
% this section, we discuss a new kind of hybrid model where $d$-faulty processes and
% crash-faulty processes occur simultaneously.
% {\tmem{Crash failures}} means a faulty processes might crash at some point,
% after which it sends no messages at all. In the hybrid model to be discussed,
% the system consists of $n$ processes, of which upto $m$ processes are
% $d$-faulty processes and upto $c$ processes might crash. An instance
% of this hybrid model is called $( n,m,d,c )$-system.
Adapt the \tmem{Interactive Consistency}  specifications to $( n,m,d,c)$-system as follows:
  \begin{itemizedot}
    \item {\tmem{Termination}}: Eventually, every process that is not $c$-faulty
     decides a value for each process in the system.
    
    \item {\tmem{Agreement}}: Every two processes that decide 
    decide the same value for each process in the system.
    
    \item {\tmem{Validity}}: If $p$ decides $x\neq \bot $ for process $q$, then
           $x$ is the initial value of $q$ and if $p$ decides $\bot$
           for process $q$, then $q$ is $c$-faulty.
  \end{itemizedot}
Remark that if $c=0$, then the problem specification is the same as
the one in Section~\ref{oralIC}.

% We focus on the situation
% where $d<n-1$, for which $d$-faulty processes always send correct messages to
% a subset of processes.
% In this section we investigate the interactive consistency problem
% for $( n,m,d,c)$-system with oral messages.


% Through this section, a $( n,m,d,c )$-system is supposed with oral
% messages. Based on the results
% for our generalized Byzantine model in the previous sections, we can have the
% following corresponding theorem.
As an extension of Theorem~\ref{consensusOral} we get:
\begin{theorem}\label{crashIC}
  Interactive Consistency can be achieved in a $( n,m,d,c )$-system with oral
  message if and only if $n> \max \{ 2m+d,2d+m \} +c$.
\end{theorem}


\iffalse
The proofs of these theorem is by induction as before. The algorithm employed
is a slightly modified version of OMIC. The existence of crash-faulty
processes might send message $\perp$ since a particular round. In order to
handle these possible empty messages, every time when we select the majority value in step $3$ of OMIC,
the count for $\perp$ is subtracted by $c$ for a $( n,m,d,c )$-system. This new
slightly different algorithm is called OMIC'. The key point is: if a process
might crash but does not crash, it behaves the same as a correct process; if a
process crash at some point, it contributes nothing to the message passing
since the counter of $\perp$ is subtracted by $c$. Due to space limitations,
the proof is moved to appendix \ref{app:thmCrash}.
\fi

% As the results in \cite{lamport1982crash,dolev1982polynomial,dwork1990knowledge} say, tolerating $c$ crash failures in the worst case
% must require at least $c+1$ rounds message-passing.
%  So we have to modify our previous algorithm to make it work.  In this section, firstly a modified version of OMIC (called OMWIC)
% is designed to reach a weak agreement, which can reduce
% the interactive consistency problem to the classical problem with only crash failures.
% Then we can use known algorithm in \cite{lamport1982crash,dolev1982polynomial,dwork1990knowledge} that only tolerates crash failures to reach interactive
% consistency.
Due to space limitations, algorithms and proofs are moved to appendix~\ref{app:thmCrash}.

%%% Local Variables: 
%%% mode: latex
%%% TeX-master: "synchrony"
%%% End: 
