%!TEX root = synchrony.tex
\section{Interactive Consistency With Oral Messages}\label{oralIC}
In this section, we establish the tight bound for a $( n,m,d )$-system to
solve \tmem{interactive consistency} with {\tmem{oral messages}}. 

\begin{theorem}
  \label{consensusOral} Interactive consistency can be achieved
  in a $( n,m,d )$-system with oral messages  if and only if $n> \max \{ 2m+d,2d+m \}$.
\end{theorem}

The proof of the theorem contains two parts with a series of lemmas. First, we
devise an algorithm we call OMIC to show the sufficiency of the theorem.
And then we show the necessary part based on scenario argument.

\subsection{Algorithm}
{\namedalgorithm{OMIC}{OMIC($0$):
\begin{enumeratenumeric}
%  \item Every process, suppose to be $i$, broadcasts its initial value, and denote the received value sent from $j$ as $v_j^i$. Let $v_i^i$ be the initial value of $i$.
    \item Every process  broadcasts its initial value.
    \item Every process receives the messages of the round.
      Let $v_j^i$ be the value received by $i$ from $j$, $v_i^i$ is the initial value of  $i$.
  %\item Set $ [ v_1^i, \ldots,  v_n^i ]$ as the output of process $i$
  \item Process $i$ decides $v_j^i$ as the initial value of process $j$.
\end{enumeratenumeric}
OMIC($k$), $k>0$:
\begin{enumeratenumeric}
  \item Each process acts as a transmitter and sends its value to other
  $n-1$ processes (receivers).
  
  \item Every receiver process uses the value it gets in Step $1$ as initial
  value, and executes OMIC($k-1$).
  
  \item After running algorithm OMIC($k-1$) in Step 2, every receiver
  process get a vector of values representing what other receivers get from the
  transmitter, and then uses the majority values as its decision on the initial
  value of the transmitter process.
\end{enumeratenumeric}}}

Just like in
{\cite{lamport1982byzantine}}, we use \tmem{transmitter} to denote the process
that wants to transmit its value, and use \tmem{receiver} to denote the process that wants to know the value of the transmitter. 

We first consider the correctness of a special case of the algorithm, and then 
the general case by induction.
%The considered special case is interesting because it shows that generally, we can get interactive consistency in two rounds.
More precisely, we show that in a $( n,m,d)$-system, two rounds are enough to achieve interactive consistency as long as 
$n \geqslant 2 ( m+d )$.
Additional rounds are necessary only when the number of processes is between $2 ( m+d ) + 1$ and $ \max \{ 2m+d,2d+m \}-1$. Beyond that, interactive consistency cannot be achieved.


\begin{lemma} \label{basicCase}
  \label{2roundlemma} If $n \geqslant 2 ( m+d )$, then the protocol OMIC($1$) achieves
  interactive consistency in $2$ rounds.
\end{lemma}

\begin{proof}
  Let us fix a process $p$ as a transmitter. If we prove that  each process decides  the
  initial value of $p$, then the lemma is proved.
  
  First suppose $p$ is correct, then all the receivers in the
  first round will get the initial value of $p$. In the second
  round when applying OMIC($0$), every receiver gets at most $m$ wrong
  values for $p$ due to the $m$ faulty processes.
As $n-1>2m$, each
  receiver decides the correct initial value of the transmitter.
  
  If $p$ is faulty, up to $d$ receivers will get wrong messages
  in the first round. Then in the second round, running OMIC($0$), every
  receiver gets at most $d+m-1$ wrong initial values for $p$,
  $m-1$ from the other faulty processes and $d$ from the receivers that have received wrong messages.
   Since $n\geqslant 2 (d+m)$, $n-1 > 2(d+m-1)$, the receiver receives a majority of the correct value and decides correctly.
\end{proof}

We now consider the case where additional rounds are necessary to achieve interactive consistency.

\begin{lemma} \label{reliableCorretness}
  \label{reliablecase}
  For any $k \geqslant 1$, if $n>2m+k$, if the transmitter is correct
 then every receiver in  OMIC($k$) decides on the initial value of the transmitter.
\end{lemma}

\begin{proof}
  The proof is by induction on $k$. Suppose the transmitter is correct. The
  lemma is trivial for $k=1$.
In the first round, all receivers get the initial value of the transmitter. In the second
  round (i.e. applying OMIC($0$)), every receiver gets at most $m$ wrong
  values for $p$ due to the $m$ faulty processes.
 As $n-1>2m$, each
  receiver decides the correct initial value of the transmitter.

  We assume now the
  lemma is true for $k-1$ ($k>1$), and we prove it for $k$.
  
%  Assume  that $n>2m+k$, then $n>2m+(k-1)$.
%  By induction hypothesis,  if the transmitter is correct
% then every receiver in  OMIC($k-1$) decides on the initial value of the transmitter.
 

%Consider now OMIC($k$).
  In the first step of the algorithm, the correct transmitter sends its value
  to the other $n-1$ receivers among which up to $m$ are faulty. 
  These receivers act as transmitters in OMIC($k-1$). 
  By induction hypothesis, if a transmitter is correct, every receiver in OMIC($k-1$)
  decides on the initial value of the transmitter.
  Then each receiver in OMIC($k$)  gets at least $n-1-m$ copies of the
  initial value in Step $3$. Since there are only $m$ faulty processes,
  $n-1-m>m+k-1 \geqslant m$, a majority of the values obtained  in Step $3$
  are the initial values. That is to say every process obtains the initial
  value of every correct process.
\end{proof}

\begin{lemma}\label{coreLemma}
  \label{mainlemma}For any $k \geqslant 1$ and $m \geqslant k$, OMIC($k$) ensures
  interactive consistency if $n>\max\{2m+k,2m+2d-k\}$.
\end{lemma}

\begin{proof}
  The proof is also by induction on $k$. If $k=1$, the condition is equal to $n>2m+2d-1$.
  Therefore by Lemma \ref{2roundlemma}, the present lemma is proved.
  Now suppose
  the lemma is correct for $k-1$. We prove it for $k$.
  
  We first assume the transmitter is correct. Since $n>2m+k$, by Lemma
  \ref{reliablecase} OMIC($k$) guarantees each process decides on  the initial
  value of the transmitter. So we have to verify the case in which
  the transmitter is faulty.
  
  If the transmitter is faulty, then at most $m-1$ of the receivers are
  faulty. Since $n-1>2m+ ( k-1 )$, and $n-1>2 ( m-1 ) +2d- (
  k-1 )$, OMIC($k-1$) ensures interactive consistency by induction hypothesis.
  Every receiver  knows the values that other receivers received in Step 1. As
  $n-1>2m+2d-k-1 \geqslant 2d$, the majority of the values the transmitter
  sent is the initial value of the transmitter. So all the processes decide
  the same correct value.
\end{proof}



\begin{lemma}\label{algICO1}
  If $m \geqslant d$ and $n>2m+d$, it is possible to achieve interactive consistency in a $(n,m,d)$-system 
  in $d+1$ rounds.
\end{lemma}

\begin{proof}
  Since $m \geqslant d$ and $n>2m+d$, we have $n>2d+d$ and $n >2m+2d-d$. Take
  $k=d$, by Lemma \ref{mainlemma} above, OMIC($d$) ensures
  interactive consistency. 
\end{proof}

\begin{lemma}\label{algICO2}
  If $d \geqslant m$ and $n>2d+m$, it is possible to achieve interactive consistency in a $(n,m,d)$-system 
 in $m+1$ rounds.
\end{lemma}

\begin{proof}
  Since $d \geqslant m$ and $n>2d+m$, we have $n>2d+m$ and $n >2m+2d-m$. Take
  $k=m$, by lemma \ref{mainlemma} above, OMIC($m$) ensures
  interactive consistency.
\end{proof}

From the above two lemmas, it is easy to deduce the sufficient part of 
Theorem~\ref{consensusOral}. 
If $m \geqslant d$ we get the result by Lemma~\ref{algICO1} and  if $m<d$ by Lemma~\ref{algICO2}.



\subsection{Impossibility}\label{oralImpossibility}

Now let us turn our attention to the necessary property of the theorem. We study two cases: $m \geqslant d$
and $d \geqslant m$. For each case, we proceed by contradiction.
We construct two scenarios such that there is a set of processes for which these two scenarios are indistinguishable. 
In the first scenario the processes of that set have to decide 0, while in the second scenario they have to decide 1.






\begin{lemma}\label{imposs:1}
  If $m \geqslant d$ and $ 2m+d \geqslant n$, it is impossible to achieve interactive consistency in a $(n,m,d)$-system with oral messages.
  \end{lemma}

Due to space limitations, the proof is moved to Appendix \ref{app:imposs:1}. Here we describe the intuition behind the proof.
  $P$ can be partitioned into three non-empty sets $A$, $B$, and $C$,
  with $| A | \leqslant m$, $| B | \leqslant m$, $| C | \leqslant d$. 
  
  We define two scenarios $\alpha$ and $\beta$.
  In $\alpha$, all processes have 0 as initial values.
  Processes in $B$ are faulty and send  to processes in $C$ messages
  pretending that processes in $A$ have 1 as initial value.
  
  In $\beta$, processes in $A$ have 1 as initial value and all  others processes have 0 as initial value.
  Processes in $A$ are faulty and send to processes in $C$ messages 
  pretending that processes in $A$ have 0 as initial value (see Figure~\ref{sc1}).
  
  In the two scenarios, processes in $C$ get the same messages, but in the first scenario 
  they have to decide 0 for processes in $A$ and $1$ in the second scenario leading to a contradiction.


\begin{lemma}\label{imposs:2}
  If $d \geqslant m$ and $2d+m \geqslant n$, it is impossible to achieve interactive consistency in a $(n,m,d)$-system with oral messages.\end{lemma}

The proof of this Lemma is similar to the above one. Due to space limitations, it is moved to Appendix \ref{app:imposs:2}.

From the above two lemmas, it is easy to deduce the necessary property of 
Theorem~\ref{consensusOral}. 
If $m \geqslant d$ we get the result by Lemma~\ref{imposs:1} and  if $m<d$ by Lemma~\ref{imposs:2}.





